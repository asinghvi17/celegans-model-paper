\RequirePackage{luatex85, shellesc} % work around the LuaTex bug
% options for packages which may be loaded elsewhere
\PassOptionsToPackage{unicode=true, pdfa}{hyperref}
\PassOptionsToPackage{hyphens}{url}
\PassOptionsToPackage{dvipsnames,svgnames*,x11names*}{xcolor}

%==============================================================================%
%                                Load packages                                 %
%==============================================================================%

\usepackage{amsmath}                  % the AMS Math package - provides useful math environments and tools

\usepackage{amssymb}                  % provides useful symbols

\usepackage{amsfonts}                 % provides nice mathematical fonts

\usepackage{bm}                       % bold math

\usepackage{calc}                     % gives the ability to calculate in the document itself, inferior to but lighter weight than pythontex

\usepackage{cancel}                   % provides environments to cancel stuff in mathmode

\usepackage{esint}                    % better and more integral signs

\usepackage{enumitem}                 % enumerate and itemize improvements

\usepackage{fancyhdr}                 % fancy headers and footers

\usepackage[vario]{fancyref}          % for fancy cross-referencing with vario-style refs

\usepackage{graphicx}                 % facilitates inclusion of external graphics

\usepackage{microtype}                % typographical improvements, kerning, etc

\usepackage{parskip}                  % produces zero \parindent and non-zero \parskip

\usepackage{setspace}                 % set space between lines

\usepackage{siunitx}                  % for the use of SI unite in mathmode

\usepackage{xcolor}                   % colours bro

\usepackage{upquote}                  % straight quotes in verbatim environments

\usepackage{unicode-math}             % allows unicode characters in mathmode

\usepackage[pdfa, unicode]{hyperref}  % hyperlinks and PDF metadata manipulation

\usepackage{tikz}                     % drawing

\usepackage{pgfplots}                 % plotting

\usepackage{geometry}                 % to set the margins of the page

%==============================================================================%
%                             Set package options                              %
%==============================================================================%

%============================%
%          AMSMath           %
%============================%

\allowdisplaybreaks

%============================%
%          Hyperref          %
%============================%

\hypersetup{
    pdftitle={A minimal reaction-diffusion neural model generates C. elegans undulation},
    pdfauthor={Anshul Singhvi},
    colorlinks=true,
    linkcolor=Maroon,
    citecolor=Blue,
    urlcolor=Blue,
    breaklinks=true,
    pdfdisplaydoctitle = true
}

%============================%
%         Microtype          %
%============================%

\UseMicrotypeSet[protrusion]{basicmath} % disable protrusion for tt fonts

%============================%
%     TikZ and PGFPlots      %
%============================%

\usetikzlibrary{
    arrows.meta,
    calc,
    decorations,
    decorations.pathreplacing,
    decorations.footprints,
    math,
    patterns,
    shadows,
    external
}

\tikzset{>=stealth}

\pgfplotsset{compat=1.16}

\usepgfplotslibrary{
    polar,
    colormaps,
    colorbrewer,
    groupplots,
    statistics
}

% This cycle list encodes the Wong colors, which are visually distinguishable.
% They also account for colorblind support, and as such are optimal for use in a paper.
\pgfplotscreateplotcyclelist{wong}{%
    color={rgb, 255 : red, 86  ; green, 180 ; blue, 233}, mark = *\\         % sky blue
    color={rgb, 255 : red, 230 ; green, 159 ; blue, 0},   mark = square*\\   % orange
    color={rgb, 255 : red, 0   ; green, 158 ; blue, 115}, mark = otimes*\\   % blueish green
    color={rgb, 255 : red, 240 ; green, 228 ; blue, 66},  mark = star\\      % yellow
    color={rgb, 255 : red, 0   ; green, 114 ; blue, 178}, mark = diamond*\\  % blue
    color={rgb, 255 : red, 213 ; green, 94  ; blue, 0},   mark = triangle*\\ % vermillion
    color={rgb, 255 : red, 204 ; green, 121 ; blue, 167}, mark = pentagon*\\ % reddish purple
}

\pgfplotsset{every axis legend/.append style={%
        cells={anchor=west}
    },
    cycle list name = wong,
}

%============================%
%          Geometry          %
%============================%

\geometry{
  a4paper,
  margin = 1.2in
}

%==============================================================================%
%                         Macros and document options                          %
%==============================================================================%

\setstretch{1} % line spacing of 1.25

\setlength{\emergencystretch}{3em}  % prevent overfull lines
\providecommand{\tightlist}{%
  \setlength{\itemsep}{0pt}\setlength{\parskip}{0pt}}
% \setcounter{secnumdepth}{0}
%
% % Redefines (sub)paragraphs to behave more like sections
% \ifx\paragraph\undefined\else
%     \let\oldparagraph\paragraph
%     \renewcommand{\paragraph}[1]{\oldparagraph{#1}\mbox{}}
% \fi
% \ifx\subparagraph\undefined\else
%     \let\oldsubparagraph\subparagraph
%     \renewcommand{\subparagraph}[1]{\oldsubparagraph{#1}\mbox{}}
% \fi

\newcommand{\inputtikz}[1]{%
  \tikzsetnextfilename{#1}%
  \input{#1.tikz}%
}

%==============================================================================%
%                            Neuron diagram macros                             %
%==============================================================================%

\newcommand{\doublec}[2]{% double diffusion arrows
  \draw  [-Circle] ($(#1.north east)!0.7!(#1.north)$) -- ($(#2.south east)!0.7!(#2.south)$);
  \draw  [Circle-] ($(#1.north west)!0.7!(#1.north)$) -- ($(#2.south west)!0.7!(#2.south)$);
  }

\newcommand{\singlec}[2]{% single diffusion arrows
  \draw [-Latex] (#1.east) -- (#2.west);
  }
